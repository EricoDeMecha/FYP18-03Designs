\addcontentsline{toc}{section}{Abstract}
\section*{Abstract}
\label{sec:}
The synthetic hydro-experimental machine used for fluid mechanics experiments in the fluids lab at Jomo Kenyatta University employs an older technology in the measurement of the dynamic pressure component in fluids. The machine is wholly mechanical as most of the parameter manipulations and measurements are done mechanically. During fluid flow experiments such as determining the coefficient of discharge for the Venturi and the orifice, one controls the flow rate by opening the gate valve in small steps by hand, reads the differential pressure from the alcohol manometers, measures the temperature, and time of the discharge simultaneously, and finally the weight of the discharge. The step size can be inconsistent since it is determined by human intuition. The initiation of time and temperature measurement is to be synchronized with discharge collection. This is of course not usually the case with this machine since it is mechanical and cannot be achieved by a human.  These limitations result in a huge error margin in the calculation of the coefficient of discharge.
\par
This project proposal seeks to reduce this error margin while maintaining the credibility of the experiment through the automation of the discharge collection unit. This will involve precisely controlling the gate valve in steps, and automating the discharge collection mechanism by techniques such as precisely collecting the discharge in steps, digitizing the temperature, time, and weight measurement of the discharge, and reducing environmental influence on these measurements.  An ergonomic user interface would also be integrated to allow for parameter manipulations and monitoring.
\par
This automation is expected to reduce the oftenly huge error margins in fluid flow experiments due to human errors.