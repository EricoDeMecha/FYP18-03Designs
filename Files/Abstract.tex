\addcontentsline{toc}{section}{Abstract}
\section*{Abstract}
\label{sec:}
\par
The synthetic hydro experimental machine used for fluid mechanics experiments in the fluids lab at JKUAT employs an older technology in the measurement of fluid properties. The machine is wholly mechanical as most of the parameter manipulations are done by hand. In experiments to determine the coefficient of discharge by using the Venturi and the Orifice, flow rate control is achieved by hand by opening the ball valve in small steps which can be inconsistent. Besides, one has to measure the temperature and time of discharge simultaneously. As a result, with these discrepancies and lack of synchronism, the findings might frequently be outside of the acceptable range due to human error.
\par
This project intends to design and fabricate an automated discharge collection process to minimize human error while maintaining the credibility of the experiment. A  modular system has been designed comprising of three major units; the discharge flow control unit, discharge handling unit and the interface and control unit. Through the use of a graphical user interface, the user will be prompted to input the time to perform the experiment and the number of steps. Based on the provided information and the pipe circumference, the system automatically determines the number of steps required. The system will automatically initiate the pump to force water into the pipe system for a specified amount of time to attain steady flow after which the ball valve closes. A servo motor attached to the valve shaft is used to open the valve in small and precise steps. A flap system diverts the discharge either into the collection tank or into the water reservoir. The collected discharge is weighed and at the same time, its temperature is measured, recorded, and displayed on the user interface after which the discharge is released into the reservoir through a solenoid valve to allow for the next step.
\par
This automation will reduce the huge error margins in fluid flow experiments due to human errors.