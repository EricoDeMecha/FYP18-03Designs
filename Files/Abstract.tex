\addcontentsline{toc}{section}{Abstract}
\section*{Abstract}
\label{sec:}
\par
The synthetic hydro-experimental machine used for fluid mechanics experiments in the fluids lab at JKUAT uses a manual mechanical system in the collection of the discharge during experiments such as the determination of coefficient of discharge of the Venturi. The user is required to turn a $1\frac{3}{4} inch $ ball valve in steps determined by human intuition, and for every step,he/she is required to slide a metallic diverter to collect the discharge to a separate tank, and at the same time, to start measuring the temperature of the discharge, and the timer using an analog stop watch. This synchronism is necessary for precise data in the computation of the fluid flow properties but cannot be achieved by human. 
\par
This project intends to design and fabricate an automated discharge collection system to minimize this human error while maintaining the credibility of the experiment. A modular system has been designed comprising of three major units; a discharge flow control unit, a discharge handling unit, and an interface and control unit. Through a touch-LCD interface, the user will set either the time interval between steps in an experiment or the number of steps for an experiment. In either case, the system will automatically determine the other based on the calibration of a full step. An MG996R servo motor has been selected to control the flow in steps that can be less than $1^{0}$ and for every step, a LA-T8 micro-linear actuator has also been selected to drive an oscillating flap through a four-bar kinematic chain. The flap diverts the flow to a discharge collection tank where its temperature and weight are measured using the selected DS18B20 temperature sensor, and four 50Kg strain type load cells connected in a WheatStone bridge respectively. The data for every step are displayed on the interface. The collected discharge is then emptied to the main reservoir through a solenoid valve before the next step.

\par
This automation will reduce the error margin in the computation of fluid flow properties due to the human error.