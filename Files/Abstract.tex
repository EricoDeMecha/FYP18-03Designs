\addcontentsline{toc}{section}{Abstract}
\section*{Abstract}
\label{sec:}
The synthetic hydro-experimental machine used for fluid mechanics experiments in the fluids lab at Jomo Kenyatta University employs an older technology in the measurement of the dynamic pressure component in fluids. The machine is wholly mechanical as most of the parameters manipulation and measurements are done mechanically. To determine the coefficient of discharge for both the Venturi and the orifice during the experiment, one has to control the flow rate by opening the gate valve in small steps in by hand, read the pressure heights from the alcohol manometers, and finally measure the temperature, time, and the weight of the discharge at the same time.  The step size can be inconsistent since it is determined by human intuition. The initiation of time and temperature measurement is to be synchronized with discharge collection. This is of course not usually the case with this machine since it is mechanical and cannot be achieved by a human. These limitations result in a huge error margin in the calculation of the coefficient of discharge.
\par
This project proposal seeks reduce this huge error margin while maintaining the credibility of the experiment through the automation of the discharge collection unit. This involves using a motor with precise steps to control the gate valves, and digitizing the temperature, time, and weight measurement of the discharge. The temperature, time, and discharge collection will be synchronized by timers controlled by logic in a micro-controller. A user interface will allow for manipulation of parameters such as the number of steps, and time for discharge collection.
\par
This automation will reduce the error margin due to human lag.
