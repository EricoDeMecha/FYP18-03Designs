\section{Introduction}
\label{sec:introduction}
\subsection{Background}
Fluid flow properties are measured using a variety of meters, including the turbine-type flow meter, the rotameter, the orifice meter, and the venturi meter \cite{pereira2009flow}. Each meter works by altering a physical property of the flowing fluid and then measuring that change. The flow is then related to the measured change \cite{miller1983flow}.
\par
The Synthetic Hydro-Experimental machine, currently installed in JKUAT, is a configurable machine with these flow meters. This machine is used to conduct experiments to establish relationships between the fluid flow properties and the behavior of the flow. It has a lift pump, gate valves, alcohol manometers, pressure gauges, a Pelton turbine, a Venturi, an orifice, and water reservoirs.  During experiments, the lift pump is turned on, and the discharge valve is fully opened to establish a steady flow.  The discharge valve is then closed. The valve is opened in small steps depending on the number of steps required. For each step, the discharge is collected, and its temperature is measured within a specific time interval. Finally, the weight of the collected discharge is also measured.

\subsection{Problem statement}

In fluid flow experiments involving the Venturi, and the orifice, the discharge steps are to be precisely opened, and the time and temperature measurements are taken simultaneously with discharge collection in order to obtain values that are within a tolerable range. This is not the case with the Synthetic Hydro-Experimental machine currently used in JKUAT since it is wholly mechanical and some of the simultaneous measurements cannot be simply achieved by a human. The flow rate is controlled by a gate valve in small steps determined by human intuition which can be inconsistent. With these inconsistencies, the results can often be outside the tolerable range. Automating the discharge collection process can minimize the error in the results and still preserve the credibility of the experiment. 


\subsection{Objectives}
\subsubsection{Main objective}

The main objective of this project is to automate the discharge collection process for the Synthetic Hydro-Experimental machine. 

\subsubsection{Specific objectives}

\begin{enumerate}
	\item To design an automated discharge flow control unit that can precisely discharge in steps.
	\item To design and fabricate a discharge collection unit with automated weight, time and temperature measurements.
    \item To design a user interface and the control algorithm.

\end{enumerate}


\subsection{Justification}

This automation will simplify the discharge collection process and ensure consistency and precision of the results obtained in each step during fluid flow experiments done on the machine. With such automation, one person can also singly complete the experiment with not much strain as opposed to the current state. The automated system will also be modular that it can be easily mounted and dismounted from the main machine with no major adjustments.